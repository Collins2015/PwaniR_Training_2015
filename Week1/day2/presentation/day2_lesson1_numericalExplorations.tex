\documentclass{beamer}
\mode<presentation>{}
\usetheme{CambridgeUS}
\usecolortheme{sidebartab}
%\pdfmapfile{+sansmathaccent.map}
\title{Numerical explorations with R}
% \subtitle{means, median, SD; use R as calc}
\author{Amos Thairu}
\usepackage{Sweave}
\begin{document}
\Sconcordance{concordance:day2_lesson1_numericalExplorations.tex:day2_lesson1_numericalExplorations.Rnw:%
1 8 1 1 0 13 1 1 2 4 0 1 2 8 1 1 2 7 0 1 2 4 1 1 2 8 0 2 2 13 0 1 2 6 1 %
2 2 13 0 1 2 5 1 1 2 7 0 2 2 8 0 1 2 3 1 1 2 8 0 2 2 9 0 1 2 12 1 1 3 8 %
0 2 2 7 0 2 2 6 0 1 1 6 0 1 2 7 1 1 2 8 0 1 2 3 1 1 2 8 0 2 2 9 0 1 2 3 %
1 1 2 10 0 1 2 2 1 1 2 6 0 1 2 6 0 1 1 5 0 1 1 6 0 1 2 3 1 1 2 4 0 2 2 %
7 0 2 2 4 0 1 2 3 1}

% \SweaveOpts{concordance=TRUE}
\begin{frame}
\titlepage
\end{frame}

% data <- read.csv('H:/PwaniStatsCourse/PwaniR_Training_2015/PwaniR_Training_2015/Week1/day2/presentation/bwmal.csv')
% bwmal_subset <- data[c(1:10,190:200),-c(1,9:12)]
% row.names(data) <- NULL
% write.csv(bwmal_subset, 'H:/PwaniStatsCourse/PwaniR_Training_2015/PwaniR_Training_2015/Week1/day2/presentation/bwmal_subset.csv')

\begin{frame}[fragile]\frametitle{Reading data files}
Import dataset
\begin{Schunk}
\begin{Sinput}
> data <- read.csv("bwmal_subset.csv")
\end{Sinput}
\end{Schunk}
\pause
\begin{itemize}
  \item Once read in, assigning the loaded data to objects
  \item Datasets in R are typically stored as data frames, which have a matrix structure 
  \item Observations are arranged as rows and variables, either numerical or categorical, are arranged as columns
  \item Dataset contains 8 demographic variables for 20 individuals
\end{itemize}
\pause
Get the dimension of the dataset
\begin{Schunk}
\begin{Sinput}
> dim(data)
\end{Sinput}
\begin{Soutput}
[1] 21  8
\end{Soutput}
\end{Schunk}
\end{frame}

\begin{frame}[fragile]\frametitle{Viewing data}
\small
Explore variable names of the dataset
\begin{Schunk}
\begin{Sinput}
> names(data) 
\end{Sinput}
\begin{Soutput}
[1] "X"        "matage"   "mheight"  "gestwks"  "sex"      "bweight"  "smoke"   
[8] "pfplacen"
\end{Soutput}
\end{Schunk}
R has ways to look at the dataset at a glance
\begin{Schunk}
\begin{Sinput}
> head(data) #Returns first six rows of dataset
\end{Sinput}
\begin{Soutput}
  X matage mheight gestwks sex bweight smoke pfplacen
1 1     26   1.575      40   0    3.11     0        0
2 2     23   1.529      40   0    2.65     0        0
3 3     18   1.540      40   1    3.41     0        0
4 4     25   1.581      40   1    2.99     0        0
5 5     25   1.555      40   1    3.16     0        0
6 6     21   1.561      40   1    2.82     0        0
\end{Soutput}
\end{Schunk}
\end{frame}

\begin{frame}[fragile]\frametitle{Viewing data}
\small
We can access variables directly by using their names, using the object \$ variable notation
 <<tidy=TRUE>>=
data$gestwks
Check last six rows of the dataset
\begin{Schunk}
\begin{Sinput}
> tail(data) #Returns first six rows of dataset
\end{Sinput}
\begin{Soutput}
     X matage mheight gestwks sex bweight smoke pfplacen
16 195     19   1.583      37   0    2.42     0        0
17 196     20   1.534      39   1    2.93     0        0
18 197     30   1.543      39   0    2.59     1        1
19 198     38   1.602      39   0    2.48     1        0
20 199     20   1.540      40   1    3.02     1        1
21 200     24   1.503      39   0    2.79     0        1
\end{Soutput}
\end{Schunk}
\end{frame}
% 
\begin{frame}[fragile]\frametitle{Viewing data}
% \small
To access a certain entry, we most commonly use object[row,column] \\
Single cell value
\begin{Schunk}
\begin{Sinput}
> data[2,3]    
\end{Sinput}
\begin{Soutput}
[1] 1.529
\end{Soutput}
\end{Schunk}
Omitting row value implies all rows; here all rows in column 3
\begin{Schunk}
\begin{Sinput}
> data[,3]
\end{Sinput}
\begin{Soutput}
 [1] 1.575 1.529 1.540 1.581 1.555 1.561 1.590 1.502 1.666 1.591 1.567 1.566
[13] 1.540 1.502 1.560 1.583 1.534 1.543 1.602 1.540 1.503
\end{Soutput}
\end{Schunk}
\end{frame}

\begin{frame}[fragile]\frametitle{More data viewing}
Omitting column values implies all columns; here all columns in row 2
\begin{Schunk}
\begin{Sinput}
> data[2,]
\end{Sinput}
\begin{Soutput}
  X matage mheight gestwks sex bweight smoke pfplacen
2 2     23   1.529      40   0    2.65     0        0
\end{Soutput}
\end{Schunk}
Can also use ranges - rows 2 and 3, columns 2 and 3
\begin{Schunk}
\begin{Sinput}
> data[2:3, 2:3]
\end{Sinput}
\begin{Soutput}
  matage mheight
2     23   1.529
3     18   1.540
\end{Soutput}
\end{Schunk}
\end{frame}

% \normal
\begin{frame}[fragile]\frametitle{Data summaries}
\begin{itemize}
  \item Enables us to see the main characteristics of data before any formal modeling or hypothesis testing
  \item Particular techniques depends on the type of variable: Continuous or categorical
  \item Continuous eg. age, height
  \item Categorical eg. smoking status, sex
\end{itemize}
\end{frame}

\begin{frame}[fragile]\frametitle{Some data explorations: Continuous variables}
\begin{Schunk}
\begin{Sinput}
> #some data explorations
> mean(data$mheight)
\end{Sinput}
\begin{Soutput}
[1] 1.558571
\end{Soutput}
\end{Schunk}
\pause
\begin{Schunk}
\begin{Sinput}
> var(data$mheight)
\end{Sinput}
\begin{Soutput}
[1] 0.001461357
\end{Soutput}
\end{Schunk}
\pause
\begin{Schunk}
\begin{Sinput}
> sd(data$matage)
\end{Sinput}
\begin{Soutput}
[1] 5.527852
\end{Soutput}
\begin{Sinput}
> median(data$matage)
\end{Sinput}
\begin{Soutput}
[1] 22
\end{Soutput}
\end{Schunk}
% \pause
% <<tidy=TRUE>>=
% median(data$matage)
% @
\end{frame}

\begin{frame}[fragile]\frametitle{More data explorations}
Produce various summaries of continous variable
\begin{Schunk}
\begin{Sinput}
> summary(data$matage)  #sumarize continous variable
\end{Sinput}
\begin{Soutput}
   Min. 1st Qu.  Median    Mean 3rd Qu.    Max. 
  17.00   20.00   22.00   23.57   25.00   38.00 
\end{Soutput}
\end{Schunk}
\end{frame}

\begin{frame}[fragile]\frametitle{Explorations for categorical variables}
Summarize single categorical variable
\begin{Schunk}
\begin{Sinput}
> table(data$sex)  
\end{Sinput}
\begin{Soutput}
 0  1 
11 10 
\end{Soutput}
\end{Schunk}
Cross-tabulation of two categorical variables
\begin{Schunk}
\begin{Sinput}
> table(data$sex,data$smoke) 
\end{Sinput}
\begin{Soutput}
    0 1
  0 8 3
  1 9 1
\end{Soutput}
\end{Schunk}
\end{frame}

\begin{frame}[fragile]\frametitle{Alternative cross tabulation}
Using \emph{'with'} command includes variable labels in the table
\begin{Schunk}
\begin{Sinput}
> with(data, table(sex,smoke)) 
\end{Sinput}
\begin{Soutput}
   smoke
sex 0 1
  0 8 3
  1 9 1
\end{Soutput}
\end{Schunk}
\end{frame}

\begin{frame}[fragile]\frametitle{Use R as calculator}
\begin{Schunk}
\begin{Sinput}
> 1000-2*10^2/(8+2)  #expression to evaluate
\end{Sinput}
\begin{Soutput}
[1] 980
\end{Soutput}
\begin{Sinput}
> #Built-in functions:
> log(1.4)  #returns the natural logarithm of the number 1.4
\end{Sinput}
\begin{Soutput}
[1] 0.3364722
\end{Soutput}
\begin{Sinput}
> log10(1.4)  # returns the log to the base of 10
\end{Sinput}
\begin{Soutput}
[1] 0.146128
\end{Soutput}
\begin{Sinput}
> sqrt(16)  #returns the square root of 16
\end{Sinput}
\begin{Soutput}
[1] 4
\end{Soutput}
\end{Schunk}
\end{frame}

\begin{frame}[fragile]\frametitle{Calculations with assignment statements}
We can store a value(s) in an R object using the assignment symbol $<-$ ("less than" followed by a hyphen)
\begin{Schunk}
\begin{Sinput}
> x <- 2.5
\end{Sinput}
\end{Schunk}
To check what is in a variable type the variable name
\begin{Schunk}
\begin{Sinput}
> x
\end{Sinput}
\begin{Soutput}
[1] 2.5
\end{Soutput}
\end{Schunk}
Can store a computation under a new R object or change the current value stored in an old object
\begin{Schunk}
\begin{Sinput}
> y <- 3*log(x)
\end{Sinput}
\end{Schunk}
\end{frame}

\end{document}
% http://www.ats.ucla.edu/stat/r/seminars/intro.htm
